
{\actuality} Информационная безопасность в современном обществе приобретает все большую актуальность, так как большое количество информации, появление современных гаджетов и развитие технологий привели к становлению постинформационного общества. Данное общество можно характеризовать как  общество потребления большого количества информации, зависимости от информационных и компьютерных технологий, а также как общество с ведущим типом коммуникации – массовой коммуникацией \cite{самойлова2016трансформации}. Человек оказывается в непрерывном удаленном взаимодействии с людьми, юридическими лицами и различным цифровыми инструментами, при этом в течении дня он вынужден многократно проходить процедуры идентификации и аутентификации в различных мобильных приложениях для совершения различных действий, будь то отправка сообщений, чтение почты или совершение платежа. При выполнении этих процедур задействованы самые разнообразные мобильные устройства, а значит задача обеспечения надежной, быстрой и удобной аутентификации в мобильных приложениях является на сегодняшний день крайне актуальной.

% замечания от В.Г. 
% весь общий анализ запихнуть во введение. В первой главе поконкретней.


%Используемые в настоящее время методики аутентификации можно подразделить на четыре категории:
%\begin{enumerate}
%\item Что пользователь знает (пароли и т.д.);
%\item что пользователь имеет (смарт карты, токены и т.д. );
%\item что есть
%\end{enumerate}



Безусловно, среди технологий аутентификации наибольшую привлекательность имеют биометрические методики. Они имеют высокую степень достоверности за счет уникальности биометрических признаков и неотделимости их от дееспособной личности. Однако, использование традиционных биометрических методик для аутентификации порою оказывается не достаточно надежными, в связи с параллельными работами над методиками их взлома \cite{богданов2017обзор}. 

Одним из путей повышение надежности аутентификации является использование комбинированных (мультимодальных) методик, сочетающих в себе технологии идентификации одновременно по нескольким признакам или категориям. Например технологии попадающие одновременно во все три категории аутентификации: <<что человек знает>>, <<что человек имеет>> и <<что есть сам человек>>. Реализация таких методик может предполагать использование дополнительных устройств. Логичнее всего в качестве таких устройств использовать что-то достаточно распространенное. В настоящее время становятся популярными умные устройства, подразумевающие повседневное ношение на запястье и осуществляющие постоянное взаимодействие с человеком - это различные модификации фитнес-браслетов и умных часов. Эти устройства разработаны, в том числе, для контроля биологических параметров и в соответствии с парадигмой интернета вещей представляют собой интерфейс <<Человек-тело-вещь>>. Объем продаж эти устройств в мире составляет десятки миллионов единиц в год.

Использование данного класса устройств совместно с традиционными смартфонами в биометрической аутентификации позволяет повысить информационную безопасность пользователей без необходимости разработки нового оборудования. При этом повышение точности аутентификации будет достигнуто за счет следующих факторов:

\begin{enumerate}
\item Наручное устройство будет являться ключом (токеном) для аутентификации в мобильном устройстве, реализуя проверку, которую можно отнести к категории <<что пользователь имеет>>. В случае кражи любого из устройств доступ злоумышленника к данным пользователя будет закрыт.
\item Наручное устройство, совместно с мобильным, будут задействованы в  аутентификации, использующей жестовую манипуляцию, и, следовательно являющуюся биометрической, относящейся к категории <<что есть сам человек>>. При этом регистрация биометрического признака (в данной работе в качестве него выбран жест) проводится датчиками двух устройств, одновременно контактирующими с разными частями тела человека -  с кистью руки и запястьем, и следовательно, учитывающими большее количество информации о биометрическом признаке по сравнению с аналогичной методикой, использующей только одно устройство. 
\item Аутентификация при помощи двух устройств использующая механизм жестовой манипуляции подразумевает, что пользователь должен знать и уметь выполнить определенный заранее придуманный жест. Следовательно такую аутентификацию можно отнести к категории <<что человек знает>>. 
\end{enumerate}

%Таким образом, задача разработки новых мтодик аутентификации с целью %повышения информационной безопасности, в условиях непрерывного %удаленного взаимодействия людей и цифровых инструментов, является на %сегодняшний день крайне актуальной.

{\aim} данной работы является повышение эффективности биометрической аутентификации при помощи жеста, совершаемого мобильным устройством. 

Для~достижения поставленной цели решались следующие {\tasks}:
\begin{enumerate}
  \item Анализ разработанных методик аутентификации пользователей в мобильных приложениях, выявление их достоинств и недостатков.
  \item Выявление класса задач, повышение эффективности решения которых может быть достигнута с использованием аутентификации при помощи жеста, совершаемого мобильным устройством (далее -- механизма жестовой манипуляции). 
  \item Разработка и исследование методов повышения надежности аутентификации за счет одновременного использования дополнительного устройства.
  \item Разработка комплекса программ аутентификации с помощью жеста.
  \item Проведение экспериментов по применению методики аутентификации для подтверждения полученных результатов, а также получение базы попыток, достаточной для получения приемлемой достоверности результатов при обработке статистических данных и последующего моделирования.
  \item Выбор наиболее эффективного с точки зрения надежности алгоритма реализации методики.
  \item Реализация разработанной методики в виде аппаратно-программного комплекса и проведение его опытной эксплуатации.
\end{enumerate}

\textbf{Объект исследования:} Биометрические системы, использующие для идентификации и аутентификации жесты, совершаемые мобильными устройствами. 

\textbf{Предмет исследования:} Методы аутентификации пользователей мобильных приложений с использующих жесты, совершаемые мобильными устройствами.

{\novelty}
\begin{enumerate}
  \item Методика биометрической аутентификации с помощью жестовой манипуляции, регистрируемая акселерометрами двух взаимодействующими друг с другом устройствами.
  \item Определение порога срабатывания как максимального расстояния от эталона, полученного выполнением серии жестовых манипуляций.
 \item Ранжирование  надежности примененного для аутентификации жеста в зависимости от суммы модулей значений ускорений.  
\end{enumerate}
 

{\influence}: 

\begin{enumerate}
	\item Было выполнено оригинальное исследование по классификации жестов пользователей, выбранных в качестве идентификаторов, на устойчивость к спуфингу.
	\item Проведено исследование, определившее равновероятный уровень ошибок первого и второго рода для девяти алгоритмов, позволившее наглядно доказать эффективность выбранного алгоритма.
	\item Разработаны мобильное приложение и аппаратно-программный комплекс, реализующий разработанную методику. 
\end{enumerate}

Результаты данной работы были результатами научно-исследовательской работы (НИР), проводимой  ФГОБУ <<Финансовый университет>> за 2017 года и признано результатом интеллектуальной деятельности данной НИР. Результаты данной работы были применены в создании аппаратно-программного комплекса АПК <<Замок МТДП>>, опытная эксплуатация которого успешно проводиться на предприятии АО <<НИИЧаспром>>.

{\methods} В ходе исследования применялись методы математической статистики, теории вероятности, теории информации, теории распознавания образов, компьютерного имитационного моделирования. 

Для компьютерного моделирования применялись программы, написанные на языке Matlab. Для реализации комплекта программ применялись языки Java и C.

{\defpositions}
\begin{enumerate}
  \item Использование двух взаимодействующих друг с другом источников регистрации жестовой манипуляции для увеличения надежности системы аутентификации.
  \item Методика суммарной оценки соответствия воспроизведенного жеста эталону.
  \item Методика ранжирования по сумме значений ускорений для классификации надежности жестовой манипуляции. 
\end{enumerate}
%В папке Documents можно ознакомиться в решением совета из Томского ГУ
%в~файле \verb+Def_positions.pdf+, где обоснованно даются рекомендации
%по~формулировкам защищаемых положений. 

{\reliability} полученных результатов подтверждена проведенными экспериментами.

% а так же обоснована и строго доказанных и корректно используемых выводах фундаментальных и прикладных наук, положения которых нашли применение в работе, корректностью постановок задач, применением строгого математического аппарата, отсутствием противоречия между результатами диссертационной работы, сделанными на их основании выводами имитационного и известными компьютерного научными моделирования, фактами, результатами апробацией основных теоретических положений в печатных трудах ВАК и докладах на отечественных и международных научных конференциях. Результаты находятся в соответствии с результатами, полученными другими авторами.

%текс хороший антиплагиат ругается

{\probation}
Основные результаты работы докладывались~на трех Российских и одной международной конференциях.

{\contribution} Автор принимал активное участие в научно-исследовательском процессе Финансового университета. Им был разработан макет для демонстрации возможностей методики, а также образец аппаратно-программного комплекса, реализующий управления запирающим устройством посредством мультимодальной аутентификации пользователя при помощи механизма жестовой манипуляции. 

%\publications\ Основные результаты по теме диссертации изложены в ХХ печатных изданиях~\cite{Sokolov,Gaidaenko,Lermontov,Management},
%Х из которых изданы в журналах, рекомендованных ВАК~\cite{Sokolov,Gaidaenko}, 
%ХХ --- в тезисах докладов~\cite{Lermontov,Management}.

\ifnumequal{\value{bibliosel}}{0}{% Встроенная реализация с загрузкой файла через движок bibtex8
    \publications\ Основные результаты по теме диссертации изложены в XX печатных изданиях,
    X из которых изданы в журналах, рекомендованных ВАК,
    X "--- в тезисах докладов.%
}{% Реализация пакетом biblatex через движок biber
%Сделана отдельная секция, чтобы не отображались в списке цитированных материалов
    \begin{refsection}[vak,wos,scopus,papers,conf]% Подсчет и нумерация авторских работ. Засчитываются только те, которые были прописаны внутри \nocite{}.
        %Чтобы сменить порядок разделов в сгрупированном списке литературы необходимо перетасовать следующие три строчки, а также команды в разделе \newcommand*{\insertbiblioauthorgrouped} в файле biblio/biblatex.tex
        \printbibliography[heading=countauthorvak, env=countauthorvak, keyword=biblioauthorvak, section=1]%
        \printbibliography[heading=countauthorwos,env=countauthorwos, keyword=biblioauthorwos, section=1]%
        \printbibliography[heading=countauthorscopus,env=countauthorscopus, keyword=biblioauthorscopus, section=1]%
	\printbibliography[heading=countauthorconf, env=countauthorconf, keyword=biblioauthorconf, section=1]%
        \printbibliography[heading=countauthorothers, env=countauthorothers, keyword=biblioauthorothers, section=1]%
        \printbibliography[heading=countauthor, env=countauthor, keyword=biblioauthor, section=1]%
        \nocite{%Порядок перечисления в этом блоке определяет порядок вывода в списке публикаций автора
                vakbib1,vakbib2,%
		wosbib1,%
		scbib1,%
                confbib1,confbib2,%
                bib1,bib2,%
        }%
	\publications\ Основные результаты по теме диссертации изложены
	\setcounter{citeauthorscwostot}{\value{citeauthorscopus}} % вместе setcounter и addtocounter добавляют пробел между словами. По-этому они так раскиданы.
        в~\arabic{citeauthor}~печатных изданиях,
	\addtocounter{citeauthorscwostot}{\value{citeauthorwos}}
	\arabic{citeauthorvak} из которых изданы в журналах, рекомендованных ВАК\sloppy
	\ifnum \value{citeauthorscwostot}>0
	, \arabic{citeauthorscwostot} "--- в~периодических научных журналах, индексируемых Web of Science и Scopus\sloppy
	\fi
	\ifnum \value{citeauthorconf}>0
	, \arabic{citeauthorconf} "--- в~тезисах докладов.
	\else
	.
	\fi
    \end{refsection}
    \begin{refsection}[vak,wos,scopus,papers,conf]%Блок, позволяющий отобрать из всех работ автора наиболее значимые, и только их вывести в автореферате, но считать в блоке выше общее число работ
        \printbibliography[heading=countauthorvak, env=countauthorvak, keyword=biblioauthorvak, section=2]%
        \printbibliography[heading=countauthorwos, env=countauthorwos, keyword=biblioauthorwos, section=2]%
        \printbibliography[heading=countauthorscopus, env=countauthorscopus, keyword=biblioauthorscopus, section=2]%
        \printbibliography[heading=countauthorothers, env=countauthorothers, keyword=biblioauthorothers, section=2]%
        \printbibliography[heading=countauthorconf, env=countauthorconf, keyword=biblioauthorconf, section=2]%
        \printbibliography[heading=countauthor, env=countauthor, keyword=biblioauthor, section=2]%
        \nocite{vakbib2}%vak
        \nocite{bib1}%other
        \nocite{confbib1}%conf
    \end{refsection}
}
%При использовании пакета \verb!biblatex! для автоматического подсчёта
%количества публикаций автора по теме диссертации, необходимо
%их~здесь перечислить с использованием команды \verb!\nocite!.
