%% Согласно ГОСТ Р 7.0.11-2011:
%% 5.3.3 В заключении диссертации излагают итоги выполненного исследования, рекомендации, перспективы дальнейшей разработки темы.
%% 9.2.3 В заключении автореферата диссертации излагают итоги данного исследования, рекомендации и перспективы дальнейшей разработки темы.
\begin{enumerate}
  \item На основе анализа механизмов жестовой манипуляции была разработана концепция методики мультимодальной биометрической аутентификации с использованием двух независимых мобильных устройств - смартфона и умных часов. 
  \item Создан макет для умных часов и смартфона, реализующий аутентификацию, а также сохраняющей данные аутентификации в виде массива временных рядов.
  \item Макет был опробован и доработан в рамках научно-исследовательских работ <<Финансового Университета>> и поставлен на баланс как результат интеллектуальной деятельности Университета. 
  \item При участи группы людей, с использованием макета, была создана база данных попыток аутентификации, составляющая более тысячи попыток.
  \item С использованием базы данных попыток аутентификации было проведено моделирование, определившее уровни равновероятного соотношения ошибок первого и второго уровня (комплексные показатели качества) для различных алгоритмов и определен наиболее удовлетворяющий с точки зрения надежности - алгоритм динамической трансформации шкалы времени с использование алгоритма Евклида для формирования матрицы расстояния.
  \item С применением выбранного алгоритма реализован аппаратно-программный комплекс <<Замок-МТДП>> - осуществляющий  аутентификацию и открытие замка при помощи механизма жестовой манипуляции применяемого для помещений или сейфовых хранилищ предприятия. 
  \item Аппаратно-программный комплекс применен для хранения материальных ценностей на предприятии АО <<НИИЧаспром>> и прошел успешную опытную эксплуатацию. Применение АПК <<Замок-МТДП>> позволило повысить надежность хранения ценностей. 
\end{enumerate}

Полученные результаты позволяют говорить об отработанности методики, и готовности внедрение ее в те сферы обеспечения информационной безопасности, где необходимо проводить срытую и надежную аутентификацию - например аутентификацию в мобильных приложениях, предполагающих работу в людных местах. 

Вторым направлением применения методики могут стать биометрические замки для различных сфер применения со сниженным уровнем энергопотребления.   